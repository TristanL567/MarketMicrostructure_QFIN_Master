%==== KNITR ===================================================================%



%==== START ===================================================================%

\documentclass{report}\usepackage[]{graphicx}\usepackage[]{xcolor}
% maxwidth is the original width if it is less than linewidth
% otherwise use linewidth (to make sure the graphics do not exceed the margin)
\makeatletter
\def\maxwidth{ %
  \ifdim\Gin@nat@width>\linewidth
    \linewidth
  \else
    \Gin@nat@width
  \fi
}
\makeatother

\definecolor{fgcolor}{rgb}{0.345, 0.345, 0.345}
\newcommand{\hlnum}[1]{\textcolor[rgb]{0.686,0.059,0.569}{#1}}%
\newcommand{\hlstr}[1]{\textcolor[rgb]{0.192,0.494,0.8}{#1}}%
\newcommand{\hlcom}[1]{\textcolor[rgb]{0.678,0.584,0.686}{\textit{#1}}}%
\newcommand{\hlopt}[1]{\textcolor[rgb]{0,0,0}{#1}}%
\newcommand{\hlstd}[1]{\textcolor[rgb]{0.345,0.345,0.345}{#1}}%
\newcommand{\hlkwa}[1]{\textcolor[rgb]{0.161,0.373,0.58}{\textbf{#1}}}%
\newcommand{\hlkwb}[1]{\textcolor[rgb]{0.69,0.353,0.396}{#1}}%
\newcommand{\hlkwc}[1]{\textcolor[rgb]{0.333,0.667,0.333}{#1}}%
\newcommand{\hlkwd}[1]{\textcolor[rgb]{0.737,0.353,0.396}{\textbf{#1}}}%
\let\hlipl\hlkwb

\usepackage{framed}
\makeatletter
\newenvironment{kframe}{%
 \def\at@end@of@kframe{}%
 \ifinner\ifhmode%
  \def\at@end@of@kframe{\end{minipage}}%
  \begin{minipage}{\columnwidth}%
 \fi\fi%
 \def\FrameCommand##1{\hskip\@totalleftmargin \hskip-\fboxsep
 \colorbox{shadecolor}{##1}\hskip-\fboxsep
     % There is no \\@totalrightmargin, so:
     \hskip-\linewidth \hskip-\@totalleftmargin \hskip\columnwidth}%
 \MakeFramed {\advance\hsize-\width
   \@totalleftmargin\z@ \linewidth\hsize
   \@setminipage}}%
 {\par\unskip\endMakeFramed%
 \at@end@of@kframe}
\makeatother

\definecolor{shadecolor}{rgb}{.97, .97, .97}
\definecolor{messagecolor}{rgb}{0, 0, 0}
\definecolor{warningcolor}{rgb}{1, 0, 1}
\definecolor{errorcolor}{rgb}{1, 0, 0}
\newenvironment{knitrout}{}{} % an empty environment to be redefined in TeX

\usepackage{alltt}

\usepackage[left=2cm, right=2cm, top=1cm, bottom=2cm]{geometry}

% Font.


% Main packages.
\usepackage[utf8]{inputenc}
\usepackage[T1]{fontenc}
\usepackage{amsmath}
\usepackage{graphicx}
\usepackage{hyperref}
\usepackage{booktabs} 
\usepackage{rotating} 
\usepackage{lmodern}

% Required for Table.


%%

\title{Market Microstructure - Documentation}
\author{Mariia, Raaif, Jan and Tristan}
\date{\today}

%==== DOCUMENT START ==========================================================%

\IfFileExists{upquote.sty}{\usepackage{upquote}}{}
\begin{document}

\maketitle

%==== ABSTRACT ================================================================%

% \begin{abstract}
% This report demonstrates the integration of R code and its output within a LaTeX document using Sweave. It covers the basic structure of a report, including a summary, chapters with subchapters, and a bibliography.
% \end{abstract}

%==== Table of content ========================================================%

\tableofcontents
\newpage

%==== Introduction ============================================================%

\chapter{Introduction}

% \section{Introduction}


%==== Chapter 1: Data =========================================================%

\chapter{Exploratory Data Analysis}

\section{Overview}

\subsection{Data Origin}

The data is sourced from Wharton Research Data Services (WRDS), specifically
the datasets relating to trade and quotes (TAQ) by the New York Stock Exchange (NYSE).
The database provides transaction information at the intraday level covering more
than 10'000 stocks listed on 16 major American exchanges. This also includes ETF-specific
data, which will be used in this analysis. Within TAQ, the datasets relating to "Consolidated Quotes" and "Consolidates Trades" were used. \\

For the analysis, the following FOMC-dates and non-FOMC dates (acting as controls)
were analysed:

\begin{table}[ht]
\centering
\caption{Analysis Dates: FOMC and Control Days (2025)}
\label{tab:fomc-dates}
\begin{tabular}{ll}
\toprule
\textbf{FOMC Date} & \textbf{Control Date} \\
\midrule
2025-01-29 & 2025-02-04 \\
2025-03-19 (Economic Projections) & 2025-03-27 \\
2025-05-07 & 2025-05-20 \\
2025-06-18 (Economic Projections) & 2025-06-23 \\
2025-07-30 & 2025-08-04 \\
2025-09-29 & 2025-10-03 \\
2025-10-29 (Economic Projections) & 2025-11-05 \\
\bottomrule
\end{tabular}
\end{table}

Within TAQ, the SPDR S\&P 500 Trust ETF (Ticker: "SPY") was analysed. The selected
time-period started from 13:00:00 and ended at 16:00:00. The first part of the period,
from 13:00:00 to 14:25:00 relates to the release of the written statement after
each meeting whilst the second period, from 14:25:00 to 16:00:00 is associated with
the speech by the chairman of the federal reserve system and the subsequent
questions and answers session.

\subsection{Data Description}

For the analysis, the following variables were downloaded as a csv-file:

\begin{table}[ht]
\centering
\caption{Variable Definitions from TAQ}
\label{tab:variables}
\begin{tabular}{ll}
\toprule
\textbf{Variable} & \textbf{Description} \\
\midrule
\multicolumn{2}{l}{\textit{Consolidated Quotes}} \\
BID & Bid price \\
BIDSIZ & Bid size (in units of trade) \\
ASK & Ask price \\
ASKSIZ & Ask size (in units oftrade) \\
\multicolumn{2}{l}{\textit{Consolidated Trades}} \\
SIZE & Volume of trade \\
PRICE & Price of trade \\
TR\_ID & Trade ID \\
\bottomrule
\end{tabular}
\end{table}

the exact time stamp is added automatically be the database.



%==== Chapter 2: Methodology ==================================================%

\chapter{Methodology}

\section{Kyle-regressions}

\section{VPIN}


%==== Chapter 3: Output & Results =============================================%

\chapter{Output}

\section{Overview}



%==== Anhang ==================================================================%

\appendix 

\chapter{Overview}

%==== END =====================================================================%

\end{document}
